\documentclass[journal,12pt,twocolumn]{IEEEtran}

\usepackage{setspace}
\usepackage{gensymb}
\singlespacing
\usepackage[cmex10]{amsmath}

\usepackage{amsthm}

\usepackage{mathrsfs}
\usepackage{txfonts}
\usepackage{stfloats}
\usepackage{bm}
\usepackage{cite}
\usepackage{cases}
\usepackage{subfig}

\usepackage{longtable}
\usepackage{multirow}

\usepackage{enumitem}
\usepackage{mathtools}
\usepackage{steinmetz}
\usepackage{tikz}
\usepackage{circuitikz}
\usepackage{verbatim}
\usepackage{tfrupee}
\usepackage[breaklinks=true]{hyperref}
\usepackage{graphicx}
\usepackage{tkz-euclide}

\usetikzlibrary{calc,math}
\usepackage{listings}
    \usepackage{color}                                            %%
    \usepackage{array}                                            %%
    \usepackage{longtable}                                        %%
    \usepackage{calc}                                             %%
    \usepackage{multirow}                                         %%
    \usepackage{hhline}                                           %%
    \usepackage{ifthen}                                           %%
    \usepackage{lscape}     
\usepackage{multicol}
\usepackage{chngcntr}
\usepackage{textcomp}
\usepackage{float}
\restylefloat{table}

\DeclareMathOperator*{\Res}{Res}

\renewcommand\thesection{\arabic{section}}
\renewcommand\thesubsection{\thesection.\arabic{subsection}}
\renewcommand\thesubsubsection{\thesubsection.\arabic{subsubsection}}

\renewcommand\thesectiondis{\arabic{section}}
\renewcommand\thesubsectiondis{\thesectiondis.\arabic{subsection}}
\renewcommand\thesubsubsectiondis{\thesubsectiondis.\arabic{subsubsection}}


\hyphenation{op-tical net-works semi-conduc-tor}
\def\inputGnumericTable{}                                 %%

\lstset{
%language=C,
frame=single, 
breaklines=true,
columns=fullflexible
}
\begin{document}

\newcommand{\BEQA}{\begin{eqnarray}}
        \newcommand{\EEQA}{\end{eqnarray}}
\newcommand{\define}{\stackrel{\triangle}{=}}
\bibliographystyle{IEEEtran}
\raggedbottom
\setlength{\parindent}{0pt}
\providecommand{\mbf}{\mathbf}
\providecommand{\pr}[1]{\ensuremath{\Pr\left(#1\right)}}
\providecommand{\qfunc}[1]{\ensuremath{Q\left(#1\right)}}
\providecommand{\sbrak}[1]{\ensuremath{{}\left[#1\right]}}
\providecommand{\lsbrak}[1]{\ensuremath{{}\left[#1\right.}}
\providecommand{\rsbrak}[1]{\ensuremath{{}\left.#1\right]}}
\providecommand{\brak}[1]{\ensuremath{\left(#1\right)}}
\providecommand{\lbrak}[1]{\ensuremath{\left(#1\right.}}
\providecommand{\rbrak}[1]{\ensuremath{\left.#1\right)}}
\providecommand{\cbrak}[1]{\ensuremath{\left\{#1\right\}}}
\providecommand{\lcbrak}[1]{\ensuremath{\left\{#1\right.}}
\providecommand{\rcbrak}[1]{\ensuremath{\left.#1\right\}}}
\theoremstyle{remark}
\newtheorem{rem}{Remark}
\newcommand{\sgn}{\mathop{\mathrm{sgn}}}
\providecommand{\abs}[1]{\vert#1\vert}
\providecommand{\res}[1]{\Res\displaylimits_{#1}}
\providecommand{\norm}[1]{\lVert#1\rVert}
%\providecommand{\norm}[1]{\lVert#1\rVert}
\providecommand{\mtx}[1]{\mathbf{#1}}
\providecommand{\mean}[1]{E[#1]}
\providecommand{\fourier}{\overset{\mathcal{F}}{ \rightleftharpoons}}
%\providecommand{\hilbert}{\overset{\mathcal{H}}{ \rightleftharpoons}}
\providecommand{\system}{\overset{\mathcal{H}}{ \longleftrightarrow}}
%\newcommand{\solution}[2]{\textbf{Solution:}{#1}}
\newcommand{\solution}{\noindent \textbf{Solution: }}
\newcommand{\cosec}{\,\text{cosec}\,}
\newcommand{\comb}[2]{{}^{#1}\mathrm{C}_{#2}}
\providecommand{\dec}[2]{\ensuremath{\overset{#1}{\underset{#2}{\gtrless}}}}
\newcommand{\myvec}[1]{\ensuremath{\begin{pmatrix}#1\end{pmatrix}}}
\newcommand{\mydet}[1]{\ensuremath{\begin{vmatrix}#1\end{vmatrix}}}
\numberwithin{equation}{subsection}
\makeatletter
\@addtoreset{figure}{problem}
\makeatother
\let\StandardTheFigure\thefigure
\let\vec\mathbf
\renewcommand{\thefigure}{\theproblem}
\def\putbox#1#2#3{\makebox[0in][l]{\makebox[#1][l]{}\raisebox{\baselineskip}[0in][0in]{\raisebox{#2}[0in][0in]{#3}}}}
\def\rightbox#1{\makebox[0in][r]{#1}}
\def\centbox#1{\makebox[0in]{#1}}
\def\topbox#1{\raisebox{-\baselineskip}[0in][0in]{#1}}
\def\midbox#1{\raisebox{-0.5\baselineskip}[0in][0in]{#1}}
\vspace{3cm}
\title{AI1103 Assignment-4}
\author{SRIVATSAN T - CS20BTECH11062}
\maketitle
\newpage
\bigskip
\renewcommand{\thefigure}{\theenumi}
\renewcommand{\thetable}{\theenumi}
Download all python codes from
\begin{lstlisting}
    https://github.com/Srivatsan-T/AI1103/tree/main/Assignment-4/codes
\end{lstlisting}
%
and latex-tikz codes from
%
\begin{lstlisting}
    https://github.com/Srivatsan-T/AI1103/blob/main/Assignment-4/Assignment-4.tex
\end{lstlisting}
\section*{QUESTION (GATE-ME-2012-45)}
A box contains 4 red balls and 6 black balls. Three balls are selected randomly from the box one after another, 
without replacement. What is the probability that the selected set contains one red ball and two black balls?
\begin{center}
    A. $\displaystyle{\frac{1}{20}}$ \hspace{0.5cm}B. $\displaystyle{\frac{1}{12}}$ \hspace{0.5cm}C. $\displaystyle{\frac{3}{10}}$ \hspace{0.5cm}D. $\displaystyle{\frac{1}{2}}$
\end{center}
\section*{SOLUTION}
This problem involves Hyper-Geometric distribution
\begin{itemize}
    \item Number of Red balls = 4
    \item Number of Black balls = 6
\end{itemize}
One can use Hyper-Geometric distribution to find out the probability that the selected set contains 1 red and 2 black balls.\\
Let M be a variable representing the number of black balls in a selection of 3 balls.\\
M has a probability mass Function:
\begin{equation}
    p_M(k) = \pr{M = k} = \frac{\comb{K}{k} \times \comb{N-K}{n-k}}{\comb{N}{n}}
\end{equation}
Here Success refers to selecting a black ball,
\begin{table}[H]
    \begin{center}
        \resizebox{\columnwidth}{!}
        {
            \begin{tabular}{|c|c|c|}
                \hline
                K & Total successes in population & 6\\
                \hline
                N & Population size & 6 + 4 = 10\\
                \hline
                k & Total observed successes & 2\\
                \hline
                n & Number of draws & 3\\
                \hline
            \end{tabular}
        }
    \end{center}
\end{table}
Probability that the selected set contains 2 black balls and 1 red ball = $\pr{M = 2}$
\begin{align}
    \pr{M = 2} = \hspace{0.2cm}& \frac{\comb{K}{2} \times \comb{N-K}{n-2}}{\displaystyle{\comb{N}{n}}}\\[0.3cm]
               = \hspace{0.2cm}& \frac{\comb{6}{2} \times \comb{10-6}{3-2}}{\displaystyle{\comb{10}{3}}}\\[0.3cm]
               = \hspace{0.2cm}& \frac{\comb{6}{2} \times \comb{4}{1}}{\displaystyle{\comb{10}{3}}}\\[0.3cm]
               = \hspace{0.2cm}& \frac{15 \times 4}{120}\\[0.3cm]
               = \hspace{0.2cm}& \frac{1}{2}
\end{align}
So the probability that the selected set of 3 balls contain 2 black balls and 1 red ball is $\displaystyle{\frac{1}{2}}$.
\begin{center}
    Correct Option : D
\end{center}
\end{document}