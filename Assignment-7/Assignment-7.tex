\documentclass[journal,12pt,twocolumn]{IEEEtran}

\usepackage{setspace}
\usepackage{gensymb}
\singlespacing
\usepackage[cmex10]{amsmath}

\usepackage{amsthm}

\usepackage{mathrsfs}
\usepackage{txfonts}
\usepackage{stfloats}
\usepackage{bm}
\usepackage{cite}
\usepackage{cases}
\usepackage{subfig}

\usepackage{longtable}
\usepackage{multirow}

\usepackage{enumitem}
\usepackage{mathtools}
\usepackage{steinmetz}
\usepackage{tikz}
\usepackage{circuitikz}
\usepackage{verbatim}
\usepackage{tfrupee}
\usepackage[breaklinks=true]{hyperref}
\usepackage{graphicx}
\usepackage{tkz-euclide}

\usetikzlibrary{calc,math}
\usepackage{listings}
    \usepackage{color}                                            %%
    \usepackage{array}                                            %%
    \usepackage{longtable}                                        %%
    \usepackage{calc}                                             %%
    \usepackage{multirow}                                         %%
    \usepackage{hhline}                                           %%
    \usepackage{ifthen}                                           %%
    \usepackage{lscape}     
\usepackage{multicol}
\usepackage{chngcntr}
\usepackage{textcomp}
\usepackage{float}
\restylefloat{table}

\DeclareMathOperator*{\Res}{Res}

\renewcommand\thesection{\arabic{section}}
\renewcommand\thesubsection{\thesection.\arabic{subsection}}
\renewcommand\thesubsubsection{\thesubsection.\arabic{subsubsection}}

\renewcommand\thesectiondis{\arabic{section}}
\renewcommand\thesubsectiondis{\thesectiondis.\arabic{subsection}}
\renewcommand\thesubsubsectiondis{\thesubsectiondis.\arabic{subsubsection}}


\hyphenation{op-tical net-works semi-conduc-tor}
\def\inputGnumericTable{}                                 %%

\lstset{
%language=C,
frame=single, 
breaklines=true,
columns=fullflexible
}
\begin{document}

\newcommand{\BEQA}{\begin{eqnarray}}
        \newcommand{\EEQA}{\end{eqnarray}}
\newcommand{\define}{\stackrel{\triangle}{=}}
\bibliographystyle{IEEEtran}
\raggedbottom
\setlength{\parindent}{0pt}
\providecommand{\mbf}{\mathbf}
\providecommand{\pr}[1]{\ensuremath{\Pr\left(#1\right)}}
\providecommand{\qfunc}[1]{\ensuremath{Q\left(#1\right)}}
\providecommand{\sbrak}[1]{\ensuremath{{}\left[#1\right]}}
\providecommand{\lsbrak}[1]{\ensuremath{{}\left[#1\right.}}
\providecommand{\rsbrak}[1]{\ensuremath{{}\left.#1\right]}}
\providecommand{\brak}[1]{\ensuremath{\left(#1\right)}}
\providecommand{\lbrak}[1]{\ensuremath{\left(#1\right.}}
\providecommand{\rbrak}[1]{\ensuremath{\left.#1\right)}}
\providecommand{\cbrak}[1]{\ensuremath{\left\{#1\right\}}}
\providecommand{\lcbrak}[1]{\ensuremath{\left\{#1\right.}}
\providecommand{\rcbrak}[1]{\ensuremath{\left.#1\right\}}}
\theoremstyle{remark}
\newtheorem{rem}{Remark}
\newcommand{\sgn}{\mathop{\mathrm{sgn}}}
\providecommand{\abs}[1]{\vert#1\vert}
\providecommand{\res}[1]{\Res\displaylimits_{#1}}
\providecommand{\norm}[1]{\lVert#1\rVert}
%\providecommand{\norm}[1]{\lVert#1\rVert}
\providecommand{\mtx}[1]{\mathbf{#1}}
\providecommand{\mean}[1]{E[#1]}
\providecommand{\fourier}{\overset{\mathcal{F}}{ \rightleftharpoons}}
%\providecommand{\hilbert}{\overset{\mathcal{H}}{ \rightleftharpoons}}
\providecommand{\system}{\overset{\mathcal{H}}{ \longleftrightarrow}}
%\newcommand{\solution}[2]{\textbf{Solution:}{#1}}
\newcommand{\solution}{\noindent \textbf{Solution: }}
\newcommand{\cosec}{\,\text{cosec}\,}
\newcommand{\comb}[2]{{}^{#1}\mathrm{C}_{#2}}
\providecommand{\dec}[2]{\ensuremath{\overset{#1}{\underset{#2}{\gtrless}}}}
\newcommand{\myvec}[1]{\ensuremath{\begin{pmatrix}#1\end{pmatrix}}}
\newcommand{\mydet}[1]{\ensuremath{\begin{vmatrix}#1\end{vmatrix}}}
\numberwithin{equation}{subsection}
\makeatletter
\@addtoreset{figure}{problem}
\makeatother
\let\StandardTheFigure\thefigure
\let\vec\mathbf
\renewcommand{\thefigure}{\theproblem}
\def\putbox#1#2#3{\makebox[0in][l]{\makebox[#1][l]{}\raisebox{\baselineskip}[0in][0in]{\raisebox{#2}[0in][0in]{#3}}}}
\def\rightbox#1{\makebox[0in][r]{#1}}
\def\centbox#1{\makebox[0in]{#1}}
\def\topbox#1{\raisebox{-\baselineskip}[0in][0in]{#1}}
\def\midbox#1{\raisebox{-0.5\baselineskip}[0in][0in]{#1}}
\vspace{3cm}
\title{AI1103 Assignment-7}
\author{SRIVATSAN T - CS20BTECH11062}
\maketitle
\newpage
\bigskip
\renewcommand{\thefigure}{\theenumi}
\renewcommand{\thetable}{\theenumi}
Download all python codes from
\begin{lstlisting}
https://github.com/CS20BTECH11062/AI1103/tree/main/Assignment-7/codes
\end{lstlisting}
%
and latex-tikz codes from
%
\begin{lstlisting}
https://github.com/CS20BTECH11062/AI1103/tree/main/Assignment-7/Assignment-7.tex
\end{lstlisting}
\section*{QUESTION (CSIR UGC NET June 2013 Q.59)}
Let $U_1,U_2\dots,U_n$ be independent and identically distributed random variables each
having a uniform distribution on (0,1). Then,$$\lim_{n \to +\infty} \pr{U_1+U_2\dots,U_n\leq \frac{3}{4}n}$$
\begin{enumerate}
    \item does not exist
    \item exists and equals 0
    \item exists and equals 1
    \item exists and equals $\frac{3}{4}$
\end{enumerate}
\section*{SOLUTION}
We use Weak law for large numbers to solve this problem. 
Let the collection of identically distributed random variables $U_1,U_2\dots,U_n$
have a finite mean $\mu$ and finite variance $\sigma^2$.
\begin{align}
    \mu = E(U_i) \hspace{0.3cm}\text{for i $\in$ (1,2,3\dots,n)}\label{0.0.1}
\end{align}
Since the distribution is uniform on (0,1), $\mu$ = 0.5. Let $M_n$ be the sample mean
\begin{align}
     M_n = \frac{U_1+U_2+U_3\dots+U_n}{n}\label{0.0.2}
\end{align}
Expected value of $M_n$ (using \eqref{0.0.2} and \eqref{0.0.1})is
\begin{align}
    E(M_n) = &\frac{E(U_1+U_2+U_3+\dots+U_n)}{E(n)}\\[0.3cm]
     = &\frac{E(U_1)+E(U_2)+\dots+E(U_n)}{n}\\
     = &\frac{n\times\mu}{n}\\
     = & \mu
\end{align}
Variance of M
\begin{align}
    Var(M_n) =& \frac{Var(U_1+U_2+U_3\dots+U_n)}{n^2}\\[0.3cm]
    =& \frac{Var(U_1) + Var(U_2)\dots+Var(U_n)}{n^2}\\
    =& \frac{n\times{\sigma^2}}{n^2}\\[0.3cm]
    =& \frac{\sigma^2}{n} \label{0.0.10}
\end{align}
From Chebyshev inequality, for any $\epsilon > 0$
\begin{align}
    \pr{|M_n-\mu|\geq \epsilon} \hspace{0.2cm} \leq \hspace{0.2cm} \frac{Var(M_n)}{\epsilon^2}
\end{align}
From \eqref{0.0.1} and \eqref{0.0.10}
\begin{align}
    \pr{\big|\frac{U_1+U_2\dots+U_n}{n} - \mu\big| \geq \epsilon} \leq \frac{\sigma^2}{n\times\epsilon^2}\notag
\end{align}
\begin{align}
    \begin{split}
    \lim_{n \to \infty} \pr{\big|\frac{U_1+U_2\dots+U_n}{n} - \mu\big| \geq \epsilon}\\
    \leq \lim_{n \to \infty} \frac{\sigma^2}{n\times\epsilon^2} \leq 0 \hspace{0.2cm} \text{for fixed $\epsilon > 0$}
    \end{split}
\end{align}
But since Probabilities are always non-negative,
\begin{align}
    \lim_{n \to \infty} \pr{\big|\frac{U_1+U_2\dots+U_n}{n} - \mu\big| \geq \epsilon} \to 0 \label{0.0.13}
\end{align}
This is known as the weak law of large numbers\\
The inverse of \eqref{0.0.13} is also true
\begin{align}
    &\lim_{n \to \infty} \pr{\big|\frac{U_1+U_2\dots+U_n}{n} - \mu\big| \leq \epsilon} \to 1 \\[0.3cm]
    &|\frac{U_1+U_2\dots+U_n}{n} - \mu\big| \leq \epsilon \hspace{0.2cm}\text{as  n $\to$ $\infty$} 
\end{align}
From $\epsilon$, n definition of limits, it is clear that 
\begin{align}
    &\frac{U_1+U_2\dots+U_n}{n} \to \mu\\
    &U_1+U_2\dots U_n \to n\times\mu \hspace{0.2cm}\text{as  n $\to$ $\infty$}
\end{align}
Since $\mu = \frac{1}{2}$,
\begin{align}
    \lim_{n \to \infty} U_1+U_2\dots U_n = \frac{1}{2}n < \frac{3}{4}n
\end{align}
So 
\begin{align}
    \lim_{n \to +\infty} \pr{U_1+U_2\dots,U_n\leq \frac{3}{4}n} = 1
\end{align}
\begin{center}
    Correct Option : C
\end{center}
\end{document}