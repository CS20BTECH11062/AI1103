\documentclass{beamer}
\usepackage{listings}
\lstset{
language=C,
frame=single, 
breaklines=true,
columns=fullflexible
}
\usepackage{subcaption}
\usepackage{url}
\usepackage{tikz}
\usepackage{graphicx}
\usepackage{multicol}
\usepackage{tkz-euclide} % loads  TikZ and tkz-base
%\usetkzobj{all}
\usetikzlibrary{calc,math}
\usepackage{float}
\usepackage{amsthm}
\newcommand\norm[1]{\left\lVert#1\right\rVert}
\renewcommand{\vec}[1]{\mathbf{#1}}
\newcommand{\R}{\mathbb{R}}
\newcommand{\C}{\mathbb{C}}
\newcommand{\comb}[2]{{}^{#1}\mathrm{C}_{#2}}
\providecommand{\brak}[1]{\ensuremath{\left(#1\right)}}
\providecommand{\abs}[1]{\vert#1\vert}
\providecommand{\fourier}{\overset{\mathcal{F}}{ \rightleftharpoons}}
\providecommand{\sbrak}[1]{\ensuremath{{}\left[#1\right]}}
\usepackage[export]{adjustbox}
\usepackage[utf8]{inputenc}
\usepackage{amsmath}
\usepackage[version=4]{mhchem}
\usetheme{Boadilla}
\title{Characteristic Function}
\author{Srivatsan T}
\institute{IITH}
\date{\today}
\begin{document}

\begin{frame}
  \titlepage
\end{frame}
\begin{frame}{Question}
  \begin{block}{UGC MATH (mathA) June 2017-Q.52}
    X and Y are independent random variables each having the density
    \begin{align}
      f(t) = \displaystyle\frac{1}{\pi} \frac{1}{1+{t}^2} -\infty < t < +\infty\notag
    \end{align}
    Then the density function of $\displaystyle\frac{X+Y}{3}$ for $-\infty <$ t $< +\infty$ is\bigskip
    \begin{multicols}{2}
      \begin{enumerate}\itemsep0.5cm
        \item $\displaystyle\frac{6}{\pi} \frac{1}{4+9{t}^2}$
        \item $\displaystyle\frac{6}{\pi} \frac{1}{9+4{t}^2}$
        \item $\displaystyle\frac{3}{\pi} \frac{1}{1+9{t}^2}$
        \item $\displaystyle\frac{3}{\pi} \frac{1}{9+{t}^2}$\\
      \end{enumerate}
    \end{multicols}
  \end{block}
\end{frame}
\begin{frame}{Characteristic Function}
  \begin{enumerate}
    \item If a random variable admits a probability density function, then the characteristic function
          is the Fourier transform of the probability density function.
    \item    It provides an alteranate way to deal with probabilities of random variables
          other than PDF and CDF.
    \item It has particularly simpler results in case of sum of independent random varaiables.
  \end{enumerate}
\end{frame}
\begin{frame}{PDF of X+Y}
  PDF of sum of random variables X and Y given their individual PDFs can be calculated using
  \begin{enumerate}
    \item Convolution
    \item Characteristic Function
  \end{enumerate}
\end{frame}
\begin{frame}{PDF of X+Y using characteristic function}
  \begin{block}{Property}
    Characteristic function of sum of independent random variables is the product of characteristic function of those random varaiables.
  \end{block}
  And we obtain the PDF of X+Y by calculating the inverse characteristic function of X+Y.
\end{frame}
\begin{frame}{CF of X and Y}
  \begin{block}{Given PDF of X and Y}
    \begin{align}
      f(t) = \displaystyle\frac{1}{\pi} \frac{1}{1+{t}^2} \hspace{0.2cm},-\infty < t < +\infty\label{0.0.1}
    \end{align}
  \end{block}
  We then calcultate the fourier transform of $f(t)$ to get the CF of X and Y.
  \begin{block}{CF of X and Y}
    \begin{align}
      g(w) =\hspace{0.2cm} & \int_{-\infty}^{\infty}  f(t) {e}^{iwt} dt                                         \\[0.2cm]
      =\hspace{0.2cm}      & \int_{-\infty}^{\infty}  \displaystyle\frac{1}{\pi} \frac{1}{1+{t}^2} {e}^{iwt} dt \\[0.2cm]
      =\hspace{0.2cm}      & e^{-\abs{w}}\hspace{0.2cm}, -\infty<w<\infty
    \end{align}
  \end{block}
\end{frame}
\begin{frame}{CF-contd.}
  Let CF of Z = X + Y be h(w)
  \begin{block}{h(w)}
    \begin{align}
      h(w) = & \hspace{0.2cm} \text{CF of x $\times$ CF of Y}  \\
      =      & \hspace{0.2cm} e^{-\abs{w}} \times e^{-\abs{w}} \\
      =      & \hspace{0.2cm} {e}^{-2\abs{w}}
    \end{align}
  \end{block}
\end{frame}

\begin{frame}{Inverse CF of Z}
  Now to find the PDF of Z, we calcultate the inverse fourier transform of h(w).
  \begin{block}{Finding $F_{X+Y}(t)$}
    \begin{align}
      F_{X+Y}(t) =\hspace{0.2cm} & \int_{-\infty}^{\infty} h(w) {e}^{-iwt} dw           \\[0.2cm]
      =\hspace{0.2cm}            & \int_{-\infty}^{\infty} {e}^{-iwt-2\abs{w}} dw       \\[0.2cm]
      =\hspace{0.2cm}            & \frac{4}{4 + {t}^2} \hspace{0.2cm}, -\infty<t<\infty
    \end{align}
  \end{block}
\end{frame}
\begin{frame}{Solution contd.}
  \begin{block}{$F_{X+Y}(t)$}
    \begin{align}
      \text{But}\hspace{0.2cm} \int_{-\infty}^{\infty} F_{X+Y}(t) dt = \int_{-\infty}^{\infty} \frac{4}{4+{t^2}} dt= 2\pi
    \end{align}
    So we plug in the normalisation factor $\displaystyle\frac{1}{2\pi}$ and the new $F_{X+Y}$ becomes
    \begin{align}
      F_{X+Y}(t) = \frac{2}{\pi} \frac{1}{4+{t}^2}\hspace{0.2cm}, -\infty<t<\infty
    \end{align}
  \end{block}
\end{frame}
\begin{frame}{Fun Approach}
  \begin{block}{Recap}
    \begin{align}
      g(w) = & \hspace{0.2cm}  \int_{-\infty}^{\infty}  f(t) {e}^{iwt} dt     \label{0.0.13} \\
      =      & \hspace{0.2cm}  e^{-\abs{w}}\hspace{0.2cm}, -\infty<w<\infty                  \\
      h(w) = & \hspace{0.2cm} {e}^{-2\abs{w}}\hspace{0.2cm}, -\infty<w<\infty
    \end{align}
  \end{block}
  Notice that $g(2w) = h(w)$\\
  Fourier transform obeys one to one correspondence
\end{frame}
\begin{frame}{Solution contd.}
  Replacing $w$ with $2w$ in \eqref{0.0.13}
  \begin{block}{$g(2w)$}
    \begin{align}
      g(2w) = & \hspace{0.2cm}  \int_{-\infty}^{\infty}  f(t) {e}^{i(2w)t} dt                                         \\
      =       & \hspace{0.2cm}  \int_{-\infty}^{\infty}  f\brak{\frac{t}{2}} {e}^{i2w\frac{t}{2}} d\brak{\frac{t}{2}} \\
      =       & \hspace{0.2cm}  \int_{-\infty}^{\infty}  \frac{1}{2} f\brak{\frac{t}{2}} {e}^{iwt} dt
    \end{align}
  \end{block}
\end{frame}
\begin{frame}{Solution contd.}
  Since $g(2w) = h(w)$
  \begin{block}{$h(w)$}
    \begin{align}
      h(w) = & \hspace{0.2cm}  \int_{-\infty}^{\infty}  \frac{1}{2} f\brak{\frac{t}{2}} {e}^{iwt} dt
    \end{align}
  \end{block}
  Let $\frac{1}{2} f\brak{\frac{t}{2}}$ be any function of t whose Characteristic function is h(w).
\end{frame}
\begin{frame}{Solution contd.}
  Recall from $\eqref{0.0.1}$ f(t) = $\displaystyle\frac{1}{\pi} \frac{1}{1+{t}^2} \hspace{0.2cm},-\infty < t < +\infty$
  \begin{block}{}
    \begin{align}
      F_Z(t) = & \hspace{0.2cm} \frac{1}{2} f\brak{\frac{t}{2}}                                                 \\
      =        & \hspace{0.2cm} \frac{1}{2} \displaystyle\frac{1}{\pi}  \frac{4}{4+{t}^2}                       \\
      =        & \hspace{0.2cm} \displaystyle\frac{1}{\pi}  \frac{2}{4+{t}^2} \hspace{0.2cm}, -\infty<t<+\infty
    \end{align}
  \end{block}
\end{frame}

\begin{frame}{$\frac{Z}{3} = \frac{X+Y}{3}$}
  We know that if a random variable M has a probability density $f_M(x)$, then the probability density of random variable kM is
  \begin{block}{}
    \begin{align}
      f_{kM}\brak{x} = \frac{1}{\abs{k}} f_M\brak{\frac{x}{\abs{k}}}
    \end{align}
  \end{block}
  Probability density of $\frac{Z}{3}$ given $F_Z(t)$ is
  \begin{block}{}
    \begin{align}
      F_{\frac{Z}{3}}(t) = & 3\times \displaystyle f_{X+Y}(3t)                                 \\
      =                    & \frac{6}{\pi} \frac{1}{4+9{t}^2}\hspace{0.2cm}, -\infty<t<+\infty
    \end{align}
    \begin{center}
      Thus option 1 is correct
    \end{center}
  \end{block}
\end{frame}
\begin{frame}{Figures}
  \begin{figure}[!]
    \centering
     \includegraphics[width = 0.8\columnwidth]{Presentation-1.png}
    \caption{PDF of X,Y and $\displaystyle\frac{X+Y}{3}$}
  \end{figure}
\end{frame}
\end{document}