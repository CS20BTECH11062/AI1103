\documentclass[journal,12pt,twocolumn]{IEEEtran}

\usepackage{setspace}
\usepackage{gensymb}
\singlespacing
\usepackage[cmex10]{amsmath}

\usepackage{amsthm}

\usepackage{mathrsfs}
\usepackage{txfonts}
\usepackage{stfloats}
\usepackage{bm}
\usepackage{cite}
\usepackage{cases}
\usepackage{subfig}

\usepackage{longtable}
\usepackage{multirow}

\usepackage{enumitem}
\usepackage{mathtools}
\usepackage{steinmetz}
\usepackage{tikz}
\usepackage{circuitikz}
\usepackage{verbatim}
\usepackage{tfrupee}
\usepackage[breaklinks=true]{hyperref}
\usepackage{graphicx}
\usepackage{tkz-euclide}

\usetikzlibrary{calc,math}
\usepackage{listings}
    \usepackage{color}                                            %%
    \usepackage{array}                                            %%
    \usepackage{longtable}                                        %%
    \usepackage{calc}                                             %%
    \usepackage{multirow}                                         %%
    \usepackage{hhline}                                           %%
    \usepackage{ifthen}                                           %%
    \usepackage{lscape}     
\usepackage{multicol}
\usepackage{chngcntr}

\DeclareMathOperator*{\Res}{Res}

\renewcommand\thesection{\arabic{section}}
\renewcommand\thesubsection{\thesection.\arabic{subsection}}
\renewcommand\thesubsubsection{\thesubsection.\arabic{subsubsection}}

\renewcommand\thesectiondis{\arabic{section}}
\renewcommand\thesubsectiondis{\thesectiondis.\arabic{subsection}}
\renewcommand\thesubsubsectiondis{\thesubsectiondis.\arabic{subsubsection}}


\hyphenation{op-tical net-works semi-conduc-tor}
\def\inputGnumericTable{}                                 %%

\lstset{
%language=C,
frame=single, 
breaklines=true,
columns=fullflexible
}
\begin{document}

\newcommand{\BEQA}{\begin{eqnarray}}
\newcommand{\EEQA}{\end{eqnarray}}
\newcommand{\define}{\stackrel{\triangle}{=}}
\bibliographystyle{IEEEtran}
\raggedbottom
\setlength{\parindent}{0pt}
\providecommand{\mbf}{\mathbf}
\providecommand{\pr}[1]{\ensuremath{\Pr\left(#1\right)}}
\providecommand{\qfunc}[1]{\ensuremath{Q\left(#1\right)}}
\providecommand{\sbrak}[1]{\ensuremath{{}\left[#1\right]}}
\providecommand{\lsbrak}[1]{\ensuremath{{}\left[#1\right.}}
\providecommand{\rsbrak}[1]{\ensuremath{{}\left.#1\right]}}
\providecommand{\brak}[1]{\ensuremath{\left(#1\right)}}
\providecommand{\lbrak}[1]{\ensuremath{\left(#1\right.}}
\providecommand{\rbrak}[1]{\ensuremath{\left.#1\right)}}
\providecommand{\cbrak}[1]{\ensuremath{\left\{#1\right\}}}
\providecommand{\lcbrak}[1]{\ensuremath{\left\{#1\right.}}
\providecommand{\rcbrak}[1]{\ensuremath{\left.#1\right\}}}
\theoremstyle{remark}
\newtheorem{rem}{Remark}
\newcommand{\sgn}{\mathop{\mathrm{sgn}}}
\providecommand{\abs}[1]{\vert#1\vert}
\providecommand{\res}[1]{\Res\displaylimits_{#1}} 
\providecommand{\norm}[1]{\lVert#1\rVert}
%\providecommand{\norm}[1]{\lVert#1\rVert}
\providecommand{\mtx}[1]{\mathbf{#1}}
\providecommand{\mean}[1]{E[#1]}
\providecommand{\fourier}{\overset{\mathcal{F}}{ \rightleftharpoons}}
%\providecommand{\hilbert}{\overset{\mathcal{H}}{ \rightleftharpoons}}
\providecommand{\system}{\overset{\mathcal{H}}{ \longleftrightarrow}}
	%\newcommand{\solution}[2]{\textbf{Solution:}{#1}}
\newcommand{\solution}{\noindent \textbf{Solution: }}
\newcommand{\cosec}{\,\text{cosec}\,}
\providecommand{\dec}[2]{\ensuremath{\overset{#1}{\underset{#2}{\gtrless}}}}
\newcommand{\myvec}[1]{\ensuremath{\begin{pmatrix}#1\end{pmatrix}}}
\newcommand{\mydet}[1]{\ensuremath{\begin{vmatrix}#1\end{vmatrix}}}
\numberwithin{equation}{subsection}
\makeatletter
\@addtoreset{figure}{problem}
\makeatother
\let\StandardTheFigure\thefigure
\let\vec\mathbf
\renewcommand{\thefigure}{\theproblem}
\def\putbox#1#2#3{\makebox[0in][l]{\makebox[#1][l]{}\raisebox{\baselineskip}[0in][0in]{\raisebox{#2}[0in][0in]{#3}}}}
     \def\rightbox#1{\makebox[0in][r]{#1}}
     \def\centbox#1{\makebox[0in]{#1}}
     \def\topbox#1{\raisebox{-\baselineskip}[0in][0in]{#1}}
     \def\midbox#1{\raisebox{-0.5\baselineskip}[0in][0in]{#1}}
\vspace{3cm}
\title{AI1103 Assignment-1}
\author{SRIVATSAN T - CS20BTECH11062}
\maketitle
\newpage
\bigskip
\renewcommand{\thefigure}{\theenumi}
\renewcommand{\thetable}{\theenumi}
Download all python codes from 
\begin{lstlisting}
https://github.com/CS20BTECH11062/AI1103/tree/main/Assignment-1/codes
\end{lstlisting}
%
and latex-tikz codes from 
%
\begin{lstlisting}
https://github.com/CS20BTECH11062/AI1103/tree/main/Assignment-1/Assignment-1.tex
\end{lstlisting}
\section*{QUESTION}
Suppose that 90 \% of people are right handed. 
What is the probability that at most 6 of a random sample 
of 10 people are right handed?\\
\section*{SOLUTION}
One can either be 'right' handed or 'left' handed. Given that 90 \% of the population is 'right' handed and since being 'right' handed and being 'left' handed are mutually exclusive,
\bigskip
\begin{itemize}
\item Probability of 'right' handed = $\Pr\brak{R}$ = \(\frac{9}{10}\)
\item Probability of 'left' handed = $\Pr\brak{L}$ = \( \frac{1}{10} \)
\end{itemize}
\bigskip
One can use binomial distribution to find out the probability that more that 6 people are 'right' handed.\\
Let X be a variable representing number of people who are 'right' handed in a given sample. Picking different number of people is an example of a Bernoulli trial.\\
So X has a binomial distribution : 
\begin{align}
\Pr\brak{X=y} = {{n}\choose{x}} * (l)^{n-y} * (r)^{y}\label{0.0.1}
\end{align}
Where
\begin{itemize}
    \item n = Total number of people = 10 
    \item l = Probability that a person is 'left' handed = \( \frac{1}{10} \)
    \item r = Probability that a person is 'right' handed = \( \frac{9}{10} \)
\end{itemize}
\bigskip
So,

\begin{align}
\Pr\brak{X=y} = {{n}\choose{y}} * \left(\frac{1}{10}\right)^{n-y} * \left(\frac{1}{10}\right)^{y}
\end{align}

Pr(at most 6 are right handed) = $\Pr\brak{X \leq 6}$ \\
$\implies$ 1 - $\Pr\brak{X \geq 7}$\\
Since\\
\bigskip
\begin{align}
\sum_{X=1}^{10} Pr(X) = 1)\label{0.0.3}
\end{align}
\bigskip
Thus, we can write:
\begin{align}
1 - \Pr\brak{X \geq 7} = 1 - \sum_{X=7}^{10} \Pr\brak{X}\label{0.0.4}
\end{align}

\begin{align}
\implies 1 - \sum_{y=7}^{10} {{10}\choose{y}} * {0.1}^{10-y} * {0.9}^{y}\label{0.0.5}
\end{align}

\begin{align}
\begin{split}
\implies 1 - {{10}\choose{7}} * {0.1}^{3} * {0.9}^{7} - {{10}\choose{8}} * {0.1}^{2} * {0.9}^{8}
\bigskip\\
-{{10}\choose{9}} * {0.1}^{1} * {0.9}^{9} - {{10}\choose{10}} * {0.1}^{0} * {0.9}^{10}\label{0.0.6}
\end{split}
\end{align}

\begin{align}
\implies \Pr\brak{X \leq 6} = 0.012795198\notag
\end{align}
\bigskip\\
Thus the probability that at most 6 people are 'right' handed out of a random sample of 10 is 0.012795198 
\end{document}