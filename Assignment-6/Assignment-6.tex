\documentclass[journal,12pt,twocolumn]{IEEEtran}

\usepackage{setspace}
\usepackage{gensymb}
\singlespacing
\usepackage[cmex10]{amsmath}

\usepackage{amsthm}

\usepackage{mathrsfs}
\usepackage{txfonts}
\usepackage{stfloats}
\usepackage{bm}
\usepackage{cite}
\usepackage{cases}
\usepackage{subfig}

\usepackage{longtable}
\usepackage{multirow}

\usepackage{enumitem}
\usepackage{mathtools}
\usepackage{steinmetz}
\usepackage{tikz}
\usepackage{circuitikz}
\usepackage{verbatim}
\usepackage{tfrupee}
\usepackage[breaklinks=true]{hyperref}
\usepackage{graphicx}
\usepackage{tkz-euclide}

\usetikzlibrary{calc,math}
\usepackage{listings}
    \usepackage{color}                                            %%
    \usepackage{array}                                            %%
    \usepackage{longtable}                                        %%
    \usepackage{calc}                                             %%
    \usepackage{multirow}                                         %%
    \usepackage{hhline}                                           %%
    \usepackage{ifthen}                                           %%
    \usepackage{lscape}     
\usepackage{multicol}
\usepackage{chngcntr}
\usepackage{textcomp}
\usepackage{float}
\restylefloat{table}

\DeclareMathOperator*{\Res}{Res}

\renewcommand\thesection{\arabic{section}}
\renewcommand\thesubsection{\thesection.\arabic{subsection}}
\renewcommand\thesubsubsection{\thesubsection.\arabic{subsubsection}}

\renewcommand\thesectiondis{\arabic{section}}
\renewcommand\thesubsectiondis{\thesectiondis.\arabic{subsection}}
\renewcommand\thesubsubsectiondis{\thesubsectiondis.\arabic{subsubsection}}


\hyphenation{op-tical net-works semi-conduc-tor}
\def\inputGnumericTable{}                                 %%

\lstset{
%language=C,
frame=single, 
breaklines=true,
columns=fullflexible
}
\begin{document}

\newcommand{\BEQA}{\begin{eqnarray}}
        \newcommand{\EEQA}{\end{eqnarray}}
\newcommand{\define}{\stackrel{\triangle}{=}}
\bibliographystyle{IEEEtran}
\raggedbottom
\setlength{\parindent}{0pt}
\providecommand{\mbf}{\mathbf}
\providecommand{\pr}[1]{\ensuremath{\Pr\left(#1\right)}}
\providecommand{\qfunc}[1]{\ensuremath{Q\left(#1\right)}}
\providecommand{\sbrak}[1]{\ensuremath{{}\left[#1\right]}}
\providecommand{\lsbrak}[1]{\ensuremath{{}\left[#1\right.}}
\providecommand{\rsbrak}[1]{\ensuremath{{}\left.#1\right]}}
\providecommand{\brak}[1]{\ensuremath{\left(#1\right)}}
\providecommand{\lbrak}[1]{\ensuremath{\left(#1\right.}}
\providecommand{\rbrak}[1]{\ensuremath{\left.#1\right)}}
\providecommand{\cbrak}[1]{\ensuremath{\left\{#1\right\}}}
\providecommand{\lcbrak}[1]{\ensuremath{\left\{#1\right.}}
\providecommand{\rcbrak}[1]{\ensuremath{\left.#1\right\}}}
\theoremstyle{remark}
\newtheorem{rem}{Remark}
\newcommand{\sgn}{\mathop{\mathrm{sgn}}}
\providecommand{\abs}[1]{\vert#1\vert}
\providecommand{\res}[1]{\Res\displaylimits_{#1}}
\providecommand{\norm}[1]{\lVert#1\rVert}
%\providecommand{\norm}[1]{\lVert#1\rVert}
\providecommand{\mtx}[1]{\mathbf{#1}}
\providecommand{\mean}[1]{E[#1]}
\providecommand{\fourier}{\overset{\mathcal{F}}{ \rightleftharpoons}}
%\providecommand{\hilbert}{\overset{\mathcal{H}}{ \rightleftharpoons}}
\providecommand{\system}{\overset{\mathcal{H}}{ \longleftrightarrow}}
%\newcommand{\solution}[2]{\textbf{Solution:}{#1}}
\newcommand{\solution}{\noindent \textbf{Solution: }}
\newcommand{\cosec}{\,\text{cosec}\,}
\newcommand{\comb}[2]{{}^{#1}\mathrm{C}_{#2}}
\providecommand{\dec}[2]{\ensuremath{\overset{#1}{\underset{#2}{\gtrless}}}}
\newcommand{\myvec}[1]{\ensuremath{\begin{pmatrix}#1\end{pmatrix}}}
\newcommand{\mydet}[1]{\ensuremath{\begin{vmatrix}#1\end{vmatrix}}}
\numberwithin{equation}{subsection}
\makeatletter
\@addtoreset{figure}{problem}
\makeatother
\let\StandardTheFigure\thefigure
\let\vec\mathbf
\renewcommand{\thefigure}{\theproblem}
\def\putbox#1#2#3{\makebox[0in][l]{\makebox[#1][l]{}\raisebox{\baselineskip}[0in][0in]{\raisebox{#2}[0in][0in]{#3}}}}
\def\rightbox#1{\makebox[0in][r]{#1}}
\def\centbox#1{\makebox[0in]{#1}}
\def\topbox#1{\raisebox{-\baselineskip}[0in][0in]{#1}}
\def\midbox#1{\raisebox{-0.5\baselineskip}[0in][0in]{#1}}
\vspace{3cm}
\title{AI1103 Assignment-6}
\author{SRIVATSAN T - CS20BTECH11062}
\maketitle
\newpage
\bigskip
\renewcommand{\thefigure}{\theenumi}
\renewcommand{\thetable}{\theenumi}
Download all python codes from
\begin{lstlisting}
https://github.com/CS20BTECH11062/AI1103/tree/main/Assignment-7/codes
\end{lstlisting}
%
and latex-tikz codes from
%
\begin{lstlisting}
https://github.com/CS20BTECH11062/AI1103/tree/main/Assignment-7/Assignment-7.tex
\end{lstlisting}
\section*{QUESTION (CSIR UGC NET June 2013 Q.59)}
The joint probability density function of (X,Y) is
\begin{equation}
    f(x,y) =
    \begin{cases}
        6(1-x) & if \hspace{0.3cm}0<y<x ,0<x<1\label{0.0.1} \\
        0      & \text{otherwise}
    \end{cases}
\end{equation}
Which among the following are correct?
\begin{enumerate}\itemsep0.3cm
    \item X and Y are not independent
    \item
          $ f_Y(y) =
              \begin{cases}
                  3\brak{y-1}^{2} & if \hspace{0.3cm}0<y<1 \\
                  0               & \text{otherwise}
              \end{cases}$
    \item X and Y are independent
    \item $ f_Y(y) =
              \begin{cases}
                  3\brak{\displaystyle{y-\frac{1}{2}{y}^2}} & if \hspace{0.3cm}0<y<1 \\
                  0                                         & \text{otherwise}
              \end{cases}$
\end{enumerate}
\section*{SOLUTION}
Given joint probability density function of X and Y, marginal probability density functions are as follows:
\begin{align}
    f_X(x) = \int_{-\infty}^{\infty} f(x,y) dy \\[0.4cm]
    f_Y(y) = \int_{-\infty}^{\infty} f(x,y) dx
\end{align}
Calculating $f_X(x)$
\begin{align}
    f_X(x) = & \int_{-\infty}^{\infty} f(x,y) dy \\
    =        & \int_{0}^{x} 6(1-x) dy            
\end{align}
\begin{align}
    f_X(x) =
    \begin{cases}
        6x(1-x) & 0<x<1     \label{0.0.6}\\
        0       & otherwise
    \end{cases}
\end{align}
Calculating $f_Y(y)$
\begin{align}
    f_Y(y) = & \int_{-\infty}^{\infty} f(x,y) dx \\
    =        & \int_{y}^{1} 6(1-x) dx\\
    =        & 6x -3{x}^2 \big|_{y}^{1}\\
    =        & 3 - 6y + 3y^{2}\\
    =        & 3{(y-1)}^{2}      
\end{align}
\begin{align}
    f_Y(y) =
    \begin{cases}
        3{(y-1)}^{2}  & 0<y<1     \label{0.0.12}\\
        0       & otherwise
    \end{cases}
\end{align}
To check whether X and Y are independent, we calculate $f_X(x) \times f_Y(y)$. From \eqref{0.0.6} and \eqref{0.0.12}
\begin{align}
    f_X(x) \times f_Y(y) = &
    \begin{cases}
        18x(1-x){(y-1)}^{2}  \\ \hspace{1.2cm}0<x<1,0<y<1\\
        0 \hspace{1cm} \text{otherwise}
    \end{cases}\\
    \neq & f(x,y)
\end{align}
Since f(x,y) and $f_X(x)\times f_Y(y)$ are different, random variables X and Y are not independent.
\begin{center}
    Options 1 and 2 are correct
\end{center}
\end{document}